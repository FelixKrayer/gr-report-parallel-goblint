%%
% This is an Overleaf template for scientific articles and reports
% using the TUM Corporate Desing https://www.tum.de/cd
%
% For further details on how to use the template, take a look at our
% GitLab repository and browse through our test documents
% https://gitlab.lrz.de/latex4ei/tum-templates.
%
% The tumarticle class is based on the KOMA-Script class scrartcl.
% If you need further customization please consult the KOMA-Script guide
% https://ctan.org/pkg/koma-script.
% Additional class options are passed down to the base class.
%
% If you encounter any bugs or undesired behaviour, please raise an issue
% in our GitLab repository
% https://gitlab.lrz.de/latex4ei/tum-templates/issues
% and provide a description and minimal working example of your problem.
%%


\documentclass[
  english,        % define the document language (english, german)
  font=times,     % define main text font (helvet, times, palatino, libertine)
  twocolumn,      % use onecolumn or twocolumn layout
]{tumarticle}


% load additional packages
\usepackage{acronym}
\usepackage{enumitem}
\usepackage{pgfplots}
\usepackage{csvsimple}
\usepackage{listings}
\usepackage{lstautogobble}

\definecolor{TUMBlue}{HTML}{0065BD}
\definecolor{TUMAccentOrange}{HTML}{E37222}
\definecolor{TUMAccentGreen}{HTML}{A2AD00}

% Settings for lstlistings
\lstset{
  numbers=left,
  %frame=single,
  basicstyle=\ttfamily,
  columns=fullflexible,
  autogobble,
  keywordstyle=\bfseries\color{TUMBlue},
  stringstyle=\color{TUMAccentGreen},
  commentstyle=\color{TUMAccentGreen},
  captionpos=b,
  basicstyle=\footnotesize,
  numbers=none
}

\newcommand{\gob} {\textsc{Goblint}}

% article metadata
\title{Parallelizing the Top-Down Solver for Faster Static Analysis}
\subtitle{Parallelisierung des top-down Solvers für schnellere statische Analyse}

\author[affil=1, email=felix.krayer@tum.de]{Felix Krayer}

\affil[mark=1]{\theUniversityName}

%TODO: add Supervisor and Advisors

\date{October 27, 2024}

\begin{document}

\maketitle



\begin{abstract}
  Abstract interpreters get slow for larger programs since solving large constraint systems through fix-point iteration is time-consuming. In this report, we aim to accelerate the static analysis of programs by parallelizing a top-down fix-point algorithm.

  We implemented three different approaches of parallel solvers in the \gob\ static analyzer. The first solver spawns multiple threads at the beginning that independently search for variables to iterate. The other two rely on a modification of the constraint system that explicitly indicates which parts can likely be solved in parallel. These two solvers differ in how they share data: one solver uses a shared data structure, while the other one employs local data structures for each thread that are merged in the end.

  Benchmarks of the three solvers compared to a baseline show, that the speedup depends on the analyzed program. This is due to how we modified the constraint system to give parallelization hints. Overall we achieved a satisfying speedup for certain programs, where parallelization hints were well-placed, with the best-performing solver reaching a speedup of around 2.0 when using two worker threads for a number of programs. Additionally, another solver provided a slight speedup for two other programs. We were not able to find a correlation between the size of the analyzed program and the speedup achieved.

  Our work provides a foundation for further research in this area, focusing on improving the parallelization hints or developing strategies to select a solver based on the program to be analyzed.
\end{abstract}


\section{Introduction}
\label{sec:introduction}
% Copied from project description.
% TODO: Adapt. E.g. remove mention parallel analysis of functions and rather explain threads (create nodes are used for that)
Static analyzers examine the source code of a program to find semantic properties without having to execute it. To gain information about a program, many analyzers compute abstract values for the different program points that over-approximate the set of possible concrete states. This is done by generating a system of constraints, where the unknowns are program points, possibly together with some calling context. The instructions in the code give rise to constraints describing how they affect the abstract state. When the system of constraints is generated, the analyzer solves it for a solution through fix-point iteration. This means that from an initial non-satisfying assignment of values to unknowns, these values are updated according to the constraints until a fix-point is reached, and all constraints are satisfied. Well-known algorithms for fix-point iteration are round-robin iteration and the work-list algorithm. Round-robin algorithms can, however, only be used for finite constraint systems and the work-list algorithm usually requires static dependencies. In contrast, the top-down solver recursively explores a possibly infinite constraint system on demand and tracks dynamic dependencies on the fly. It aims to find solutions for a defined set of interesting unknowns, e.g., an end node of the main program. For that, it recursively evaluates stable values for unknowns that are needed to compute the values for the set of interesting ones. If the value for a stable unknown is needed, it can just be looked up and does not have to be calculated again. However, if a new value for an unknown is computed and updated during the solving process, it is necessary to destabilize all unknowns that depend on this updated unknown. This means that the values for the now destabilized unknowns have to be evaluated again as they have to factor in the new value of the updated unknown. These dependencies of unknowns are detected dynamically during the solving process and are saved in a data structure. Further data structures are used to keep track of stable unknowns and their values. 
The time of analysis grows together with program size. For example, an analysis of SQL lite 3 takes multiple hours with the \gob\ static analyzer. A significant part of the overall analysis time is spent by the solver on finding a solution for the constraint system. Thus, improving the performance of a solver with respect to computation time seems a promising way to improve the speed of the whole analysis. An idea to approach this issue is to equip the top-down solver with a way to execute tasks in parallel and find steps of the solving process, where a speedup can likely be achieved through parallelization. For example, an unknown can depend on multiple other unknowns, e.g., when at a given point in the program, it is not known which of two functions is executed and thus both functions have to be analyzed. In this case, one can imagine, that the recursive computation of a stable value for the unknowns corresponding to the different functions could be parallelized, and the computation time could be reduced.
In our research, we want to investigate existing approaches in related work and gain potential insights into the parallelization of solvers for constraint systems.
As our main contribution, we aim to implement a parallelized top-down solver for the \gob\ analyzer.
% TODO: Add chapter references
For this, the first step is to protect the existing data structures used by the solver to make them thread-safe and thus avoid race conditions. We allow the right-hand sides of the constraints to indicate which unknowns will certainly be queried in the evaluation, as parallelization is only reasonable for those. Furthermore, we want to determine locations in the program where parallelization of the solver can be utilized and implement it for some of those instances. The solver is extended with a thread pool to allow multiple instances of it to work in parallel. We also have to identify critical sections of the solver, where synchronization is necessary, as well as establish communication between the threads, for example when unknowns have to be destabilized. This has to happen when the value of an unknown is no longer stable and might change in future iterations.
Lastly, we evaluate our implementation using the SV-Comp benchmark suite. As metrics, we use the number of timeouts and precise verdicts, as we hope to avoid some timeouts and thus find more correct solutions through the implemented parallelization. Furthermore, we also evaluate the efficiency by considering the analysis time per program.

\section{Background}
\label{sec:background}
%TODO: cite Gob homepage?
\gob\ is a static analyzer for C programs for detecting bugs and errors. Since it computes an over-approximation of the program behavior, it is also suited to show correctness. Thus, if \gob\ concludes that a program does not contain certain bugs, we can be sure that this holds. To prove certain properties about the program to be analyzed, it computes abstract states for each program point. These states describe all possible concrete state, the program can be in at that point in a sound manner. \gob\ finds the abstract states by solving a constraint system over the program points. This constraint system is generated from the \ac{cfg} according to the specifications of the type of analysis to be performed. The solution of the constraint system is a mapping from program points to abstract states, from which properties about bugs, errors and correctness can be deduced. \gob\ uses a fix-point solver to compute a satisfying solution.

  \subsection{Side-effecting constraint systems}
  \label{sec:background:constrSys}
  In this section we describe the constraint systems as they are generated and solved in \gob\ in more detail. The variables of the constraint system correspond to program points of the \ac{cfg}. If the program is analyzed context sensitively, each variable additionally has a context from an arbitrary set of contexts and thus, multiple variables can exist for the same program point with different contexts. This allows to analyze function calls from different calling contexts separately for increased precision. However, the additional context potentially makes the space of variables for the constraint system infinitely large as the context domain might be infinite. Even though this sounds bad at first, since a non-trivial solution to an infinite constraint system cannot be computed in finite time, this is not an issue. In the case of program analysis, the solution to the constraint system does not have to include a satisfying mapping to all variables, but only to those reachable from a set of one or more \textit{interesting} variables. In the case of program analysis in \gob\ this is usually the variable corresponding to the program point at the end of the main function.
  The value domain of the constraint system is a set of abstract states, that varies according to the specification of the analysis to be performed. In any case it has to be a complete lattice with a well-defined partial order, meet and join operations as well as a largest (top) and a least (bottom) value. Furthermore, to ensure termination of the constraint solver, widen and narrow operations are required.
  The constraint system now defines a constraint function for each variable. We refer to this function as the ``\ac{rhs}'' of a variable. The constraint system requires the variable to hold a value that is greater or equal to the value computed by the \ac{rhs} with respect to the partial order of the value lattice.
  The \ac{rhs} of a function can depend on other variables in the constraint system. We say, it \textit{queries} another variable for its value and uses that for its calculations.
  A peculiarity of the \gob\ analyzer is that it uses \textit{side-effecting} constraint systems. This means that some \ac{rhs} trigger \textit{side-effects} to other variables. When a side-effect is triggered, it posts a value to another variable in the constraint system, i.e., the variable that receives the side-effect has to hold a value, that is greater or equal to its previous value and the value received through the side effect. In \gob\ this is for example used for function calls to post the state from the caller to the variable at the start of the called function.
  For \gob\ it is ensured, that all variables that receive side-effects only have a trivial \ac{rhs}, i.e., one that is only a constant initial value.
  We can think of the constraint system as a graph, where the program points are the nodes and the edges represent queries and side-effects from the \ac{rhs}, i.e., if the \ac{rhs} of variable $x$ queries variable $y$, we draw a \texttt{query} edge from $x$ to $y$. Similarly, a side-effect from $x$ to $y$ results in an \texttt{side-effect} edge from $x$ to $y$. The \texttt{query} edges span a graph very similar to the \ac{cfg} with flipped edges, since the \ac{rhs} for a variable usually queries the variable corresponding to the program point before it and applies the abstract effect of the instruction between the two program points to the queried value.
  We note here, that certain \acp{rhs} in \gob\ query a variable but discard the result, i.e., the value from the queried value is not used in the calculation of this \ac{rhs}. This is done to force that a satisfying solution for this and all variables it depends on has to be calculated. Furthermore, this forces side-effects from the \acp{rhs} of these variables to be triggered at some point. Concretely, queries whose results are discarded are used in the constraints for thread-creations. When a thread is created at the \ac{cfg} edge from variable $x$ to $y$, the \ac{rhs} of $y$ queries not only $x$ but also the return-variable that corresponds to the program point, where the function of the created thread returns. The value of this return-variable is not used to compute the \ac{rhs}-value of $y$, but it forces the computation of a satisfying solution for the variables corresponding to the thread.
  We can define a sub-system of the constraint system originating at the return-variable of the created thread. Besides the return variable, the sub-system of the thread contains all variables that are queried directly or indirectly in the \ac{rhs} evaluation of the return-variable. In general, this sub-system is disjunct from the rest of the constraint-system with respect to \texttt{query} edges. However, there is a special case if a function is called from a created thread and another location in the program. If additionally the contexts of these calls are identical, two or more sub-systems can overlap in the variables of the function call with the same context. In the case of context-insensitive analysis, the contexts are always identical. Note that in any case side-effects happen across sub-systems.


  \subsection{Top-down solver}
  \label{sec:background:td}
  % TODO: cite "three improvements to top down solver" ?
  % TODO: somehow integrate notion of poi variables earlier
  \gob\ mainly employs \acp{td} to solve its constraint systems. The most optimized of these implements many improvements and additional features, including but not limited to solving with widening and narrowing phases and incremental analysis. 
  In this section we introduce a simplified version of this solver that is the basis for the parallelized solvers we introduce in~\autoref{sec:method:td_parallel}.
  The task of the solver is to compute a solution to the constraint system. Since the domain of the constraints is a complete lattice, this can be achieved through fix-point iteration. This means that all variables are assigned the bottom value of the lattice, after which the variables are repeatedly \textit{iterated} until a fix-point is reached. In each iteration of a variable, its \ac{rhs} is evaluated for a new value. If the new value is not the same as their current one, the variable is assigned a value greater than or equal to both the current and the new value. Once all variables have a that is at least as great as their \ac{rhs}, a fix-point is reached.
  Different solvers employ different strategies to decide which variable should be iterated next. Before explaining the strategy of the \ac{td} we introduce some important properties the solver tracks for the constraint variables:
  \begin{itemize}[leftmargin=*]
    \item \textbf{value:} The current value assigned to this variable from the value domain
    \item \textbf{called:} A called variable is currently in the stack of variables to be iterated.
    \item \textbf{stable:} A stable variable currently satisfies its constraints. This holds as long as none of the variables it depends on change their value.
    \item \textbf{influences:} A transitive relation between two variables $x$ and $y$. ``$x$ influences $y$'' means that the value of variable $y$ depends on the value of $x$, because the \ac{rhs} of $y$ queries $x$.
    \item \textbf{wpoint:} If a variable is updated while it is a widening-point (wpoint), widening is applied when computing the new value. This is necessary to ensure termination of the solver.
  \end{itemize}
  The \ac{td} starts from one variable form the set of interesting variables, for which it is supposed to compute a value and starts iterating it. An iteration of a variable means, first setting it called and stable, then evaluating its \ac{rhs} and updating its value if necessary and lastly removing it from the set of called variables. If during the evaluation of the \ac{rhs} of a variable $x$, a value from another variable $y$ is queried that is neither called nor stable, the iteration of $x$ is paused and $y$ is iterated until a stable value is found for it. If $y$ is called or stable, the query just returns its current value which is then used in the \ac{rhs} evaluation of $x$. Furthermore, for such a query it is tracked that $y$ influences $x$. This results in the solver recursively following the \texttt{query} edges of the constraint graph until the solver reaches a variable, whose \ac{rhs} does not query any other variable, or a variable already marked as \textit{called}. From this variable, the solver backtracks the \texttt{query} edge. In each backtracking step, the current value of the queried variable is used in the \ac{rhs} evaluation of the querying variable. The querying variable is then updated with the result of the \ac{rhs} before taking the next backtracking step. When the backtracking reaches the initial variable from which the solving process was started, this variable is stable. This process is repeated with the other variables from the set of interesting variables until all have a stable solution.
  When the value of a variable was changed during an iteration, it is the variables that depend on this changed variable have to be destabilized. This is necessary as these variables need to take the new value of the changed variable into account. The destabilization is done by recursively following the influences-relation of the updated variable and setting all encountered variables to \textit{not stable} while removing the used paths from the influences-relation to avoid infinite loops during destabilization. Destabilizing is necessary because the influenced variables have to be iterated again, since their \ac{rhs} might change as it queries the changed variable.
  If the solver encounters a \textit{called} variable while exploring the \texttt{query} edges, this variable is marked as a wpoint. This happens for program loops, since the variable at the loop head queries the last variable inside the loop next to the one before the loop. A variable being a wpoint means, that when this variable has to be updated during an iteration, i.e., the value from the \ac{rhs} is not equal to the current value, widening is applied. This means that the current value is not joined with the new one, but widened by it, which ensures termination of the solver. 
  Lastly, we note that side-effects are computed as soon as they are encountered during the evaluation of an \ac{rhs}. If a variable receiving a side-effect has to be updated, all variables influenced by it are destabilized in the same manner as described above.
  % TODO: solver pseudo code?
\section{Methodology}
\label{sec:method}

  \subsection{Constraint systems with \texttt{create} edges}
  \label{sec:method:create}
  TODO

  \subsection{Lockable Hashtable}
  \label{sec:method:LHM}
  TODO

  \subsection{Parallelized top-down solvers}
  \label{sec:method:td_parallel}

    \subsubsection{Stealing TD}
    \label{sec:method:td_parallel:stealing}
    TODO

    \subsubsection{TD2}
    \label{sec:method:td_parallel:base}
    TODO

    \subsubsection{Dist TD}
    \label{sec:method:td_parallel:dist}
    TODO

\section{Evaluation}
\label{sec:eval}
We want to evaluate and compare our implementations of the three different parallel solvers in \gob. Before comparing their performance with respect to evaluation time, we investigate if the parallelization results in a loss of precision.
The source repository of \gob\ contains more than 1400 regression tests. These are small programs focussed on testing various edge cases of the different features of the analyzer. We observe that for less than 0.5\% of the regression tests the parallelized solvers from~\autoref{sec:method:td_parallel} were less precise than the single-threaded solver we introduced in~\autoref{sec:background:td}. This loss of precision was non-deterministic, i.e., in some runs there was no loss of precision observed. Furthermore, when analyzing real-world programs like we do in the following sections, no loss of precision was observed in the output. Thus, we believe that the parallelized solvers can be considered as precise as the single-threaded solver we compared them against. A precision loss was only observed in a few regression tests specialized in testing edge cases. We note here that \gob\ typically uses an improved version of the single-threaded \ac{td} from this report and thus is even more precise in 32 of the regression tests.

  \subsection{Analysis speed}
  \label{sec:eval:speed}
  To evaluate the speed of the different solvers, we analyze several programs and compare the time of the analysis. We use two groups of programs for this evaluation. The first are a selection of the GNU core utility programs (coreutils)~\cite{gnuCoreutils}. Before analysis these programs were combined, i.e., a single source code file was produced, that contains the original program together with the dependencies of other included files. The exact code files used lie in the \gob\ benchmark repository~\cite{goblintBench} that contains benchmark programs for the analyzed.
  The second group are programs from the pthread folder of the \gob\ benchmark repository. These are programs that use threads more extensively than the coreutil programs, i.e., they use threads slightly more often and the spawned threads perform much more work. We included these programs, since the shared memory \ac{td} and the disjunct \ac{td} depend on threads being spawned in the analyzed program to add tasks that are solved in parallel.\\
  For comparison, we analyze each program once with the single-threaded \ac{td} from~\autoref{sec:background:td} and twice with each of the parallelized \acp{td} from~\autoref{sec:method:td_parallel}, where one run was done with only the main thread working and one run with two worker threads. We do this, so we can compare the analysis time between the single-threaded \ac{td} and a certain parallelized \ac{td} in more detail, e.g., attribute differences in analysis time to the inherently different algorithms of the solvers or to the parallelization.

  \begin{figure}
    \begin{tabular}{l|l}
      cksum & 1.0s
    \end{tabular}
    \caption[Eval results]{Eval results}
    \label{fig:results}
  \end{figure}

  When investigating the results of the analysis from~\autoref{fig:results}
\section{Related Work}
\label{sec:relatedWork}
% move after Introduction?

TODO

\section{Conclusions}
\label{sec:conclusions}
In order to explore the potential of speeding up abstract interpretation by parallelizing the fix-point computation, we implemented three approaches in the \gob\ analyzer and evaluated them with respect to analysis time.
We introduced \texttt{create} edges to the side-effecting constraint systems used in \gob\ to let \acp{rhs} indicate where parallelized solving is reasonable. Furthermore, we adapted the framework of the analyzer, such that the parallelized solvers can run without concurrency issues and implemented a lockable hash-table to allow for concurrent updates on that data structure. As our main contribution, we implemented three concepts of parallelized \acl{td} in \gob\ and evaluated them on different types of programs.\\
From the Evaluations in~\autoref{sec:eval} we conclude, that the three parallelized solvers can be considered as precise as the single-threaded solver in most cases. The speedup of parallelized solving depends a lot on the program to be analyzed and the selected solver. The stealing \ac{td} was the only solver that was able to achieve a noticeable speedup on two of the coreutil programs, that only use multithreading to a minimal extent. In contrast to that, the shared memory \ac{td} and the disjunct \ac{td} perform worse on these programs but provide noticeable speedup on the pthread group of programs. This is to be expected however, since these two solvers depend on the \texttt{create} edges we placed at the creation of threads. As for a comparison between these two solvers, we saw the disjunct \ac{td} performing better than the shared memory \ac{td} in general.

  \label{sec:conclusions:futureWork} 
  \subsection{Future Work}
  We think that we have not yet exhausted the potential of parallelization. Besides thread creation, there are other locations in the constraint system, where parallelization can potentially be useful, e.g., when at a function call it is not certain which function is called and both have to be evaluated in parallel. Thus, we want to introduce a different kind of \texttt{create} edge at such positions. In contrast to the existing one, this \texttt{create} indicates that the value of the target variable is queried at a late point.
  Furthermore, we want to investigate the programs more closely, where certain solvers achieve a speedup. We hope to identify certain characteristics that we can use to automatically decide which solver is likely best suited for analyzing this program quickly.

\section*{Abbreviations}
\begin{acronym}
  \acro{rhs}[\texttt{rhs}]{right-hand side}
  \acro{cfg}[CFG]{control-flow graph}
  \acro{td}[TD]{top-down solver}
  \acro{prio}[prio]{priority}
\end{acronym}

\newpage
\bibliographystyle{splncs04}
\bibliography{bibliography.bib}

\newpage
\onecolumn
\section*{Appendix}
%TODO: refine figure
\begin{table}[h]
  \begin{tabular}{l|r|r|r|r|r|r|r}%
    \bfseries Program & \bfseries baseline & \bfseries stealing & \bfseries stealing & \bfseries shared memory & \bfseries shared memory & \bfseries disjunct & \bfseries disjunct\\
    & & \bfseries 1 thread & \bfseries 2 threads & \bfseries 1 thread & \bfseries 2 threads & \bfseries 1 thread & \bfseries 2 threads
    \csvreader[head to column names]{resources/index.csv}{}
    {\\\hline\name&\default&\pstealingone&\pstealingtwo&\pbaseone&\pbasetwo&\pdistonesg&\pdisttwosg}
  \end{tabular}
  \caption{Timing values in seconds from the benchmarks. The stealing solver with two threads timeouted for the knot program at 900s.}
  \label{fig:timing_all}
\end{table}

\end{document}
