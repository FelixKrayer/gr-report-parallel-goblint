%%
% This is an Overleaf template for scientific articles and reports
% using the TUM Corporate Desing https://www.tum.de/cd
%
% For further details on how to use the template, take a look at our
% GitLab repository and browse through our test documents
% https://gitlab.lrz.de/latex4ei/tum-templates.
%
% The tumarticle class is based on the KOMA-Script class scrartcl.
% If you need further customization please consult the KOMA-Script guide
% https://ctan.org/pkg/koma-script.
% Additional class options are passed down to the base class.
%
% If you encounter any bugs or undesired behaviour, please raise an issue
% in our GitLab repository
% https://gitlab.lrz.de/latex4ei/tum-templates/issues
% and provide a description and minimal working example of your problem.
%%


\documentclass[
  english,        % define the document language (english, german)
  font=times,     % define main text font (helvet, times, palatino, libertine)
  twocolumn,      % use onecolumn or twocolumn layout
]{tumarticle}


% load additional packages
\usepackage{acronym}
\usepackage{enumitem}
\usepackage{pgfplots}
\usepackage{csvsimple}
\usepackage{listings}
\usepackage{lstautogobble}

\definecolor{TUMBlue}{HTML}{0065BD}
\definecolor{TUMAccentOrange}{HTML}{E37222}
\definecolor{TUMAccentGreen}{HTML}{A2AD00}

% Settings for lstlistings
\lstset{
  numbers=left,
  %frame=single,
  basicstyle=\ttfamily,
  columns=fullflexible,
  autogobble,
  keywordstyle=\bfseries\color{TUMBlue},
  stringstyle=\color{TUMAccentGreen},
  commentstyle=\color{TUMAccentGreen},
  captionpos=b,
  basicstyle=\footnotesize,
  numbers=none
}

\newcommand{\gob} {\textsc{Goblint}}

% article metadata
\title{Parallelizing the Top-Down Solver for Faster Static Analysis}
\subtitle{Parallelisierung des top-down Solvers für schnellere statische Analyse}

\author[affil=1, email=felix.krayer@tum.de]{Felix Krayer}

\affil[mark=1]{\theUniversityName}

%TODO: add Supervisor and Advisors

\date{October 27, 2024}

\begin{document}

\maketitle



\begin{abstract}
  Abstract interpreters get slow for larger programs since solving large constraint systems through fix-point iteration is time-consuming. In this report, we aim to accelerate the static analysis of programs by parallelizing a top-down fix-point algorithm.

  We implemented three different approaches of parallel solvers in the \gob\ static analyzer. The first solver spawns multiple threads at the beginning that independently search for variables to iterate. The other two rely on a modification of the constraint system that explicitly indicates which parts can likely be solved in parallel. These two solvers differ in how they share data: one solver uses a shared data structure, while the other one employs local data structures for each thread that are merged in the end.

  Benchmarks of the three solvers compared to a baseline show, that the speedup depends on the analyzed program. This is due to how we modified the constraint system to give parallelization hints. Overall we achieved a satisfying speedup for certain programs, where parallelization hints were well-placed, with the best-performing solver reaching a speedup of around 2.0 when using two worker threads for a number of programs. Additionally, another solver provided a slight speedup for two other programs. We were not able to find a correlation between the size of the analyzed program and the speedup achieved.

  Our work provides a foundation for further research in this area, focusing on improving the parallelization hints or developing strategies to select a solver based on the program to be analyzed.
\end{abstract}


\section{Introduction}
\label{sec:introduction}
% Copied and slightly adapted from project description.
Static analyzers examine the source code of a program to find semantic properties without having to execute it. To gain information about a program, many analyzers compute abstract values for the different program points that over-approximate the set of possible concrete states. This is done by generating a system of constraints, where the unknowns are program points, possibly together with some calling context. The instructions in the code give rise to constraints describing how they affect the abstract state. When the system of constraints is generated, the analyzer solves it for a solution through fix-point iteration. This means that from an initial non-satisfying assignment of values to unknowns, these values are updated according to the constraints until a fix-point is reached, and all constraints are satisfied. Well-known algorithms for fix-point iteration are round-robin iteration and the work-list algorithm. Round-robin algorithms can, however, only be used for finite constraint systems and the work-list algorithm usually requires static dependencies. In contrast, the top-down solver recursively explores a possibly infinite constraint system on demand and tracks dynamic dependencies on the fly. It aims to find solutions for a defined set of interesting unknowns, e.g., an end node of the main program. For that, it recursively evaluates stable values for unknowns that are needed to compute the values for the set of interesting ones. If the value for a stable unknown is needed, it can just be looked up and does not have to be calculated again. However, if a new value for an unknown is computed and updated during the solving process, it is necessary to destabilize all unknowns that depend on this updated unknown. This means that the values for the now destabilized unknowns have to be evaluated again as they have to factor in the new value of the updated unknown. These dependencies of unknowns are detected dynamically during the solving process and are saved in a data structure. Further data structures are used to keep track of stable unknowns and their values. 
The time of analysis grows together with program size. For example, an analysis of SQL lite 3 takes multiple hours with the \gob\ static analyzer. A significant part of the overall analysis time is spent by the solver on finding a solution for the constraint system. Thus, improving the performance of a solver with respect to computation time seems a promising way to improve the speed of the whole analysis. An idea to approach this issue is to equip the top-down solver with a way to execute tasks in parallel and find steps of the solving process, where a speedup can likely be achieved through parallelization. For example, an unknown can depend on multiple other unknowns, e.g., when a thread is created and the thread function has to be analyzed as well as the rest of the program. In this case, one can imagine, that stable values for the unknowns corresponding to the thread function could be computed parallel to the unknowns corresponding to the main program, and the computation time could be reduced.
As our main contribution, we aim to implement a parallelized top-down solver for the \gob\ analyzer.

This report is structured as follows: In~\autoref{sec:background} we introduce side-effecting constraint systems as they are used in \gob\ and explain how a single-threaded top-down solver works. In~\autoref{sec:method} we present our contributions. We explain, how we adapt the constraint system to indicate for which unknowns parallelized solving can be useful. Furthermore, we implement a thread-safe data structure in this section. In~\autoref{sec:method:td_parallel} we present three different concepts of parallelized solvers and implement them in \gob\ as our main contribution. We evaluate our implementation by comparing the runtimes of the solvers in~\autoref{sec:eval}. Lastly we give our conclusions and ideas for future work in~\autoref{sec:conclusions} 

\section{Background}
\label{sec:background}
%TODO: cite Gob homepage?
\gob\ is a static analyzer for C programs for detecting bugs and errors but also showing correctness. To prove certain properties about the program to be analyzed, it computes abstract states for each program point. These states describe all possible concrete state, the program can be in at that point in a sound manner. \gob\ finds the abstract states by solving a constraint system over the program points. This constraint system is generated from the control-flow graph according to the specifications of the type of analysis to be performed. The solution of the constraint system is a mapping from program points to abstract states, from which properties about bugs, errors and correctness can be deduced. \gob\ uses a fix-point solver to compute a satisfying solution.

  \subsection{Side-effecting constraint systems}
  \label{sec:background:constrSys}
  In this section we describe the constraint systems as they are generated and solved in \gob\ in more detail. The variables of the constraint system are pairs of one program point and the corresponding calling context to allow the analysis of function calls in different contexts. However, for simplicity we think of the variables being just program points without context in this report. %TODO: maybe remove or reword
  The value domain of the constraint system is a space of abstract states, that varies according to the specification of the analysis to be performed. In any case it has to be a complete lattice with a well-defined partial order, meet and join operations as well as a largest (top) and a least (bottom) value. Furthermore, to ensure termination of the constraint solver, widen and narrow operations are required.
  The constraint system now defines a constraint function for each variable. We refer to this function as the ``right-hand side (\texttt{rhs})'' of a variable. The constraint system requires the variable to hold a value that is greater or equal to the value computed by the \texttt{rhs} with respect to the partial order of the value lattice.
  The \texttt{rhs} of a function can depend on other variables in the constraint system. We say, it \textit{queries} another variable for its value and uses that for its calculations.
  A peculiarity of the \gob\ analyzer is that it uses \textit{side-effecting} constraint systems. This means that some \texttt{rhs} trigger \textit{side-effects} to other variables. When a side-effect is triggered, it posts a value to another variable in the constraint system, i.e., the variable that receives the side-effect has to hold a value, that is greater or equal to its previous value and the value received through the side effect. In \gob\ this is for example used for function calls to post the state from the caller to the variable at the start of the called function.
  For \gob\ it is ensured, that all variables that receive side-effects only have a trivial \texttt{rhs}, i.e., one that is only a constant initial value.
  % Leave out?
  We can think of the constraint system as a graph, where the program points are the nodes and the edges represent queries and side-effects from the \texttt{rhs}, i.e., if the \texttt{rhs} of variable $x$ queries variable $y$, we draw a \textit{query} edge from $x$ to $y$. Similarly a side-effect from $x$ to $y$ results in an \textit{side-effect} edge from $x$ to $y$. The \textit{query} edges span a graph very similar to the control-flow graph with flipped edges, since the \texttt{rhs} for a variable usually queries the variable corresponding to the program point before it and applies the abstract effect of the instruction between the two program points to the queried value.

  \subsection{Top-down solver}
  \label{sec:background:td}
  % TODO: cite "three improvements to top down solver" ?
  % TODO: somehow integrate notion of poi variables earlier
  \gob\ mainly employs top-down solvers to solve its constraint systems. The most optimized of these uses implements many improvements and additional features, including but not limited to solving with widening and narrowing phases and incremental analysis. 
  In this section we introduce a simplified version of this solver that is the basis for the parallelized solvers we introduce in~\autoref{sec:method:td_parallel}.
  The task of the solver is to compute a solution to the constraint system. Since the domain of the constraints is a complete lattice, this can be achieved through fix-point iteration. This means that all variables are assigned the bottom value of the lattice, after which the variables are repeatedly \textit{iterated} until a fix-point is reached. This means that their \texttt{rhs} is evaluated for a new value. If the new value is not the same as their current one, the variable is assigned a value greater than or equal to both the current and the new value. Once all variables have a that is at least as great as their \texttt{rhs}, a fix-point is reached.
  Different solvers employ different strategies to decide which variable should be iterated next. The top-down solver starts from one or more interesting variables, for which it is supposed to compute a value. For \gob\ this is usually the variable corresponding to the program point at the end of the main function. It then recursively follows the \textit{query} edges of the constraint graph and marking all queried variables as \textit{called}. Furthermore, for each \textit{query} edge from variable $x$ to $y$, the solver follows, it tracks, that the variable $x$ depends on the value of variable $x$. Once the solver reaches a variable, whose \texttt{rhs} does not query any other variable, or it encounters a variable already marked as \textit{called} it starts iterating. After a variable is iterated, it is marked as \textit{stable}. A \textit{stable} variable currently satisfies all constraints, as long as none of the variables it depends on change their value.  

  % TODO: solver pseudo code?

\section{Methodology}
\label{sec:method}

  \subsection{Constraint systems with \texttt{create} edges}
  \label{sec:method:create}
  TODO

  \subsection{Lockable Hashtable}
  \label{sec:method:LHM}
  TODO

  \subsection{Parallelized top-down solvers}
  \label{sec:method:td_parallel}

    \subsubsection{Stealing TD}
    \label{sec:method:td_parallel:stealing}
    TODO

    \subsubsection{TD2}
    \label{sec:method:td_parallel:base}
    TODO

    \subsubsection{Dist TD}
    \label{sec:method:td_parallel:dist}
    TODO

\section{Evaluation}
\label{sec:eval}
We want to evaluate and compare our implementations of the three different parallel solvers in \gob. Before comparing their performance with respect to evaluation time, we investigate if the parallelization results in a loss of precision.
The source repository of \gob\ contains more than 1400 regression tests. These are small programs focussed on testing various edge cases of the different features of the analyzer. We observe that for less than 0.5\% of the regression tests the parallelized solvers from~\autoref{sec:method:td_parallel} were less precise than the single-threaded solver we introduced in~\autoref{sec:background:td}. This loss of precision was non-deterministic, i.e., in some runs there was no loss of precision observed. Furthermore, when analyzing real-world programs like we do in the following sections, no loss of precision was observed in the output. Thus, we believe that the parallelized solvers can be considered as precise as the single-threaded solver we compared them against. A precision loss was only observed in a few regression tests specialized in testing edge cases. We note here that \gob\ typically uses an improved version of the single-threaded \ac{td} from this report and thus is even more precise in 32 of the regression tests.

  \subsection{Analysis speed}
  \label{sec:eval:speed}
  To evaluate the speed of the different solvers, we analyze several programs and compare the time of the analysis. We use two groups of programs for this evaluation. The first are a selection of the GNU core utility programs (coreutils)~\cite{gnuCoreutils}. Before analysis, these programs were combined, i.e., a single source code file was produced, that contains the original program together with the dependencies of other included files. The exact code files used lie in the \gob\ benchmark repository~\cite{goblintBench} that contains benchmark programs for the analyzed.
  The second group are programs from the pthread folder of the \gob\ benchmark repository. These are programs that use threads more extensively than the coreutil programs, i.e., they use threads slightly more often and the spawned threads perform much more work. We included these programs, since the shared memory \ac{td} and the disjunct \ac{td} depend on threads being spawned in the analyzed program to add tasks that are solved in parallel.\\
  For comparison, we analyze each program once with the single-threaded \ac{td} from~\autoref{sec:background:td} as a baseline and twice with each of the parallelized \acp{td} from~\autoref{sec:method:td_parallel}, where one run was done with only the main thread working and one run with two worker threads. We do this, so we can compare the analysis time between the single-threaded \ac{td} and a certain parallelized \ac{td} in more detail, e.g., attribute differences in analysis time to the inherently different algorithms of the solvers or to the parallelization. Since the programs are different in size and general analysis time, we mainly focus on the \textit{speedup} as a metric. We calculate the speedup of the parallelized solvers with the following formula:
  \begin{equation*}
    \text{speedup(parallel solver)} = \frac{\text{time(baseline solver)}}{\text{time(parallel solver)}}
  \end{equation*}
  With this we get a value larger than 1, if the parallelized solver is faster and a value smaller than 1 if it is slower. The raw timing values are listed in the appendix in~\autoref{fig:timing_all}.

  \begin{figure}
    \includegraphics[width=0.5\textwidth]{../resources/Stealing_draft.png}
    \caption[Speedup of the stealing \ac{td}]{Speedup of the stealing \ac{td}. The runs with a single worker thread are shown in blue, while runs with two threads are represented by orange bars. The red bar marks a timeout in a run with 2 threads}
    \label{fig:speedup_stealing}
    % TODO: list programs
  \end{figure}

  When investigating the results for the stealing \ac{td} from~\autoref{fig:speedup_stealing}, we notice, that in general this solver is slower or just as fast as the baseline. Furthermore, the difference between running this solver single threaded or with two worker threads is minimal. Interesting are the programs \texttt{df.c} and \texttt{ls.c} where the stealing \ac{td} achieved a speedup of 1.2 and 1.3 respectively. However, for the program \texttt{cp.c} it was significantly slower than the baseline. Unfortunately we are not able to explain the timeout for the \texttt{knot.c} program, as it terminated in a reasonable time when analyzed with a debugging configuration.

  \begin{figure}
    \includegraphics[width=0.5\textwidth]{../resources/SharedMem_draft.png}
    \caption[Speedup of the shared memory \ac{td}]{Speedup of the shared memory \ac{td}. The runs with a single worker thread are shown in blue, while runs with two threads are represented by orange bars}
    \label{fig:speedup_shared_mem}
    % TODO: list programs
  \end{figure}

  \autoref{fig:speedup_shared_mem} shows the speedup results for the shared memory \ac{td}. We see this solver being significantly slower for all of the coreutil programs, with the double threaded configuration only outperforming the single threaded one in three cases. However, the pthread group of programs paint a different picture. As expected, the single threaded configuration is slower than the baseline for these programs. When allowing the shared memory \ac{td} to use a second worker thread for parallel work, it achieves significant speedup for four of the six programs compared to the baseline. This speedup even reaches 1.4 for two of these.

  \begin{figure}
    \includegraphics[width=0.5\textwidth]{../resources/Disjunct_draft.png}
    \caption[Speedup of the disjunct \ac{td}]{Speedup of the disjunct \ac{td}. The runs with a single worker thread are shown in blue, while runs with two threads are represented by orange bars}
    \label{fig:speedup_disjunct}
    % TODO: list programs
  \end{figure}

  We see the disjunct \ac{td} keeping up with the baseline for most of the coreutil programs when it is run with two worker threads. Similar to the shared memory \ac{td}, it performs better on the pthread programs, where it reaches a comparatively high speedup of around 1.8 for three programs.

  %TODO: exact timing values are in appendix
\section{Related Work}
\label{sec:relatedWork}
% move after Introduction?

TODO

\section{Conclusions}
\label{sec:conclusions}
TODO
\section{Outlook}
\label{sec:outlook}
TODO
- more create edges (e.g. unknown function)

\section*{Abbreviations}
\begin{acronym}
  \acro{rhs}[\texttt{rhs}]{right-hand side}
  \acro{cfg}[CFG]{control-flow graph}
  \acro{td}[TD]{top-down solver}
  \acro{prio}[prio]{priority}
\end{acronym}

\newpage
\bibliographystyle{splncs04}
\bibliography{bibliography.bib}

\newpage
\onecolumn
\section*{Appendix}
%TODO: refine figure
\begin{table}[h]
  \begin{tabular}{l|r|r|r|r|r|r|r}%
    \bfseries Program & \bfseries baseline & \bfseries stealing & \bfseries stealing & \bfseries shared memory & \bfseries shared memory & \bfseries disjunct & \bfseries disjunct\\
    & & \bfseries 1 thread & \bfseries 2 threads & \bfseries 1 thread & \bfseries 2 threads & \bfseries 1 thread & \bfseries 2 threads
    \csvreader[head to column names]{resources/index.csv}{}
    {\\\hline\name&\default&\pstealingone&\pstealingtwo&\pbaseone&\pbasetwo&\pdistonesg&\pdisttwosg}
  \end{tabular}
  \caption{Timing values in seconds from the benchmarks. The stealing solver with two threads timeouted for the knot program at 900s.}
  \label{fig:timing_all}
\end{table}

\end{document}
